\documentclass[a4paper, 12pt]{article}
\usepackage[utf8]{inputenc}
\usepackage[brazil]{babel}
\usepackage{amsmath}
\usepackage{geometry}
\geometry{a4paper, margin=1in}

\title{Conversor de HTML para React com Validação Automatizada Usando Agentes Artificiais}
\author{}
\date{}

\begin{document}

\maketitle

\begin{abstract}
Este projeto visa a entrega de conversor eficiente e robusto para transformar código HTML em código React. O desenvolvimento será conduzido de forma incremental, organizado em sprints, seguindo o conceito de desenvolvimento ágil.

O projeto se concentrará inicialmente na conversão básica de elementos HTML simples, evoluindo para suportar todos os aspectos da gramática HTML, incluindo atributos, estilos inline, eventos e componentes aninhados. O código Python será desenvolvivo, otimizado e documentado em português do Brasil seguindo as melhores práticas de desenvolvimento.

Além do conversor serão desenvolvidos agentes artificiais capazes de validar a qualidade do código React gerado. Esses agentes utilizarão modelagem matemática baseada em lógica clássica e fuzzy, e implementarão aprendizado por reforço, permitindo que melhorem continuamente seu desempenho com base nos feedbacks de analistas humanos.

O projeto será bem documentado e testado, garantindo sua corretude, escalabilidade e usabilidade.

Nota: Este é um esboco do projet, sofrerá alteracoes!. 
\end{abstract}

\end{document}
